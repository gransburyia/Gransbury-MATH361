% --------------------------------------------------------------
% This is all preamble stuff that you don't have to worry about.
% Head down to where it says "Start here"
% --------------------------------------------------------------
 
\documentclass[12pt]{article}
 
\usepackage[margin=1in]{geometry} 
\usepackage{amsmath,amsthm,amssymb}
\usepackage{ dsfont }
\usepackage{ amssymb }
\usepackage{ textcomp }
 
\newcommand{\N}{\mathbb{N}}
\newcommand{\Z}{\mathbb{Z}}
 
\newenvironment{theorem}[2][Theorem]{\begin{trivlist}
\item[\hskip \labelsep {\bfseries #1}\hskip \labelsep {\bfseries #2.}]}{\end{trivlist}}
\newenvironment{lemma}[2][Lemma]{\begin{trivlist}
\item[\hskip \labelsep {\bfseries #1}\hskip \labelsep {\bfseries #2.}]}{\end{trivlist}}
\newenvironment{exercise}[2][Exercise]{\begin{trivlist}
\item[\hskip \labelsep {\bfseries #1}\hskip \labelsep {\bfseries #2.}]}{\end{trivlist}}
\newenvironment{problem}[2][Problem]{\begin{trivlist}
\item[\hskip \labelsep {\bfseries #1}\hskip \labelsep {\bfseries #2.}]}{\end{trivlist}}
\newenvironment{question}[2][Question]{\begin{trivlist}
\item[\hskip \labelsep {\bfseries #1}\hskip \labelsep {\bfseries #2.}]}{\end{trivlist}}
\newenvironment{corollary}[2][Corollary]{\begin{trivlist}
\item[\hskip \labelsep {\bfseries #1}\hskip \labelsep {\bfseries #2.}]}{\end{trivlist}}
 
\begin{document}
 
% --------------------------------------------------------------
%                         Start here
% --------------------------------------------------------------
 
\title{PartialSum Gransbury}%replace X with the appropriate number
\author{Isabella Gransbury\\ %replace with your name
MATH 361} %if necessary, replace with your course title
 
\maketitle
 
\begin{problem}{1}The partial sum I created is:
	\begin{align*}
	\sum_{i=1}^{n} \ i^2 + i! \ / \ (i-1)!.
	\end{align*}
\end{problem}

\begin{problem}{2}Converge or Diverge?\\
	Product 1: I believe the sum will diverge because at 45 terms the product steadily increases towards positive infinity.\\
	Product 2: I believe the sum will converge to 1.01106503. At 45 terms the last 15 converge to 1.01106503.\\
	Product 3: I believe the sum will diverge because at 45 terms the product steadily increases towards positive infinity.
\end{problem}

\end{document}